\documentclass[]{article}
\usepackage[top=0.6in, bottom=0.6in, left=0.6in, right=0.6in]{geometry}
\usepackage[pdftex]{graphicx}
\usepackage[T2A]{fontenc}
\usepackage[utf8]{inputenc}
\usepackage[russian, english]{babel}
\usepackage{enumerate}

\selectlanguage{russian}

\begin{document}
	\begin{enumerate}[1. ]
		\item Сюжет
		\item Механики
		\item Цены действий
	\end{enumerate}
	\newpage
	
	\large{\textbf{Сюжет:}}
	\par Мы играем за студента, которому нужно \textbf{ВЫЖИТЬ}

	\
	
	\
	
	\large{\textbf{Механики:}}
	\begin{enumerate}[- ]
		\item У нас есть 3 основные шкалы, которые нужно поддерживать: голод, бодрость и настроение. А также счетчик денег и шкала успеваемости как необходимые для геймплея, но не основные для игры.
		\item Игроку доступы действия, благодаря которым игрок восстанавливает шкалы, но при этом чем больше выгода от действия, тем больше будут сбоасываться другие шкалы (например, некое действие "Подработка" будет увеличивать количество денег, но при этом будет тратиться шкала бодрости)
		\item Действия будут находится в отдельных менюшках около шкал, и действия в каждой из менюшек будут соответствовать шкалам (например около менюшки "Бодрость" будут находится действия на восстановление бодрости (вздремнуть, покемарить, поспать))
		\item Также действия будут затрачивать время (к примеру, пусть игровой день - это 30 очков, действие "Здоровый сон" будет отнимать от этих очков 10. В итоге, когда очки кончатся, начнется новый день и очки возобновятся)
		\item С течением игрового времени будет падать шкала успеваемости, для её восполнения требуется ходить на пары, которые будут тратить большое количество бодрости, голода, настроения.
		\item В некоторых временных промежутках, будут некоторые испытания (самосты, кантроша, экзамен), для которых обязателен определенный уровень успеваемости. Если же это не будет возможным для выполнения, то можно дать игроку некоторый период времени на исправление, иначе можно наложить некий штраф на игрока, или счетчик выговоров, после 3 которых игрок потеряет прогресс
		\item для прогрессии, после проигрыша нужно сохранять количество дней, чтоб можно было хвастаться
		\ 
		
		\
		
		\textbf{из необязательных, но интересных механик:}
		\item при некоторых действияхбудет тратится некая сумма денег, в качестве оплаты метро. Можно купить проездной, но для этого через некоторые временные промежутки его нужно будет пополнять
		\item скины (ну как же без этого)
		\item достижения
		\item отдельная шкала выполнения опт
	\end{enumerate}
	
	\
	
	\
	
	\large{\textbf{Цены действий:}}
	\begin{enumerate}[-]
		\item Действия бодрости будут восполнять сон, более дорогие действия еще будут подымать настроение, но голод будет снижаться
		\item пары снимают все, но случайное количетво
	\end{enumerate}
\end{document}