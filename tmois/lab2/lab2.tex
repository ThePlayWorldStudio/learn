\documentclass[]{article}
\usepackage[top=0.6in, bottom=0.6in, left=0.6in, right=0.6in]{geometry}
\usepackage[pdftex]{graphicx}
\usepackage[T2A]{fontenc}
\usepackage[utf8]{inputenc}
\usepackage[russian, english]{babel}
\usepackage{enumerate}

\selectlanguage{russian}

\begin{document}
	\begin{center}
		\par Белорусский государственный университет информатики и радиотехники
	\end{center}
	\vspace*{10cm}
	\begin{center}
		\huge
		\par \textbf{Лабораторная работа №2}\\
		\vspace{0.5cm}
		\normalsize 
		"Алгоритмы операций над графиками и их реализация"
	\end{center}
	\vspace*{8cm}
	\begin{flushright}
		\large
		Подготовил:\\
		Рассохов Е. П., гр. 421703
	\end{flushright}
	\newpage
	
	
	\large{\textbf{Постановка задачи:}}
	\par Даны два графика. Найти их инверсию и композицию. Графики задаются перечислением.\\
	
	\large{\textbf{Уточнение постановка задачи:}}
	\begin{enumerate}[1.]
		\item Элементами графиков могут быть любые числа выбраные пользователем.
		\item Мощность может быть представлена натуральными числами.
		\item Операция выбирается путем ввода цифры.
	\end{enumerate}
	\large{\textbf{Определения :}}
	\begin{itemize} 
		\item График - это множество пар, т. е. иножество, каждый элемент которого является парой или кортежом длины 
	\end{itemize}
	\large{\textbf{Алгоритм: }}
	\begin{enumerate}[1.]
		\item \textbf{Пользователь вводит график 1:} 
		\begin{enumerate}[{1}. 1.]
			\item Пользователь вводит 
		\end{enumerate}
	\end{enumerate}
\end{document}