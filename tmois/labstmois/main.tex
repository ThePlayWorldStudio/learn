\documentclass[]{article}
\usepackage[top=0.6in, bottom=0.6in, left=0.6in, right=0.6in]{geometry}
\usepackage[pdftex]{graphicx}
\usepackage[T2A]{fontenc}
\usepackage[utf8]{inputenc}
\usepackage[russian, english]{babel}
\usepackage{enumerate}

\selectlanguage{russian}

\begin{document}
	\begin{center}
		\par Белорусский государственный университет информатики и радиотехники
	\end{center}
	\vspace*{10cm}
	\begin{center}
		\huge
		\par \textbf{Лабораторная работа №1}\\
		\vspace{0.5cm}
		\normalsize 
		"Множества. Операции над множествами"
	\end{center}
	\vspace*{8cm}
	\begin{flushright}
		\large
		Подготовил:\\
		Рассохов Е. П., гр. 421703
	\end{flushright}
	\newpage
	
	
	\large{\textbf{Постановка задачи:}}
	\par Даны два множества. Найти их объединение, пересечение, разность, симметричная разность, дополнение. Множества задаются перечислением.\\
	
	\large{\textbf{Уточнение постановка задачи:}}
	\begin{enumerate}[1.]
		\item Элементами множества могут быть любые целые числа, относящиеся к промежутку значений, выбраных пользователем.
		\item мощность может быть представлена натуральными числами.
		\item пользователь сам выбирает как ему задать множество.
		\item все операции выполняются по очереди
	\end{enumerate}
	\large{\textbf{Определения :}}
	\begin{itemize}
		\item Множество - это любая определенная совокупность объектов. Объекты, из которых составлено множество, называются его элементами. Элементы множества различны и отличны друг от друга.
		\item Мощность множества - для конечных множеств мощность - это число элементов множества.
		\item Объединение множеств - это множество, которое содержит в себе все элементы исходных множеств.
		\item Пересечение множеств - это множество, состоящее из элементов, которые одновременно принадлежат исходным множествам.
		\item Разность множеств – множество, в которое входят все элементы первого множества, не входящие во второе множество.
		\item Симметрическая разность — множество, включающее все элементы исходных множеств, не принадлежащие одновременно обоим исходным множествам.
		\item Множество A’ называется дополнением множества A до некоторого универсального множества U, если оно состоит из элементов, принадлежащих множеству U и не принадлежащих множеству A.
	\end{itemize}
	\large{\textbf{Алгоритм: }}
	\begin{enumerate}[1.]
		\item \textbf{Пользователь вводит универсальное множество} \begin{enumerate}[{1}. 1.]
			 \item Пользователь вводит начальное значение универсума
			 \item Пользователь вводит конечное значение универсума
		\end{enumerate}
		\item \textbf{Создание универсального множества}
		\begin{enumerate}[{2}. 1.] 
		\item Создаем вектор
		\item Записываем в него начальное значение
		\item Увеличиваем начальное значение на 1 и добавляем его в вектор
		\item Повторяем пункт 2. 3. до тех пор, пока увеличиваемое значение не станет равным конечному
		\end{enumerate}
		\item \textbf{Создание множества А}
		\begin{enumerate}[{3}. 1.] 
		\item Пользователь вводит мощность для множества А
		\item Создаем массив А
		\item Если мощность не равна нулю, переходим к пункту 3. 4. , иначе переходим к пункту 4.
		\item Пользователь вводит элемент множества
		\item Если введенный элемент меньше конечного значения универсума и больше начального универсума переходим к пункту 3. 6. , иначе выводим текст с ошибкой ввода и возвращаемся к пункту 3. 4.
		\item Проходимся по вписаным ранее элементам массива. Если новый элемент не равен предыдущим вписаным добавляем новый элемент в массив, иначе выводим текст с ошибкой ввода и возвращаемся к пункту 3. 4.
		\item Повторяем пукты с 3. 4. по 3. 6. до тех пор, пока массив А полностью не заполнится
		\end{enumerate}
		\item \textbf{Создание массива B}
		\begin{enumerate}[{4}. 1.]
		\item Повторяем пункты с 3. 1. по 3. 7. только для массива B
		\end{enumerate}
		\item \textbf{Сортировка массива А и массива В}
		\begin{enumerate}[{5}. 1.]
		\item Проверяем 2 соседних элемента массива. Если 1 элемент больше 2 меняем их местами, иначе переходим к следующим двум элементам.
		\item Повторяем пункт 5. 1.  до тех пор, пока массив не отсортируется
		\end{enumerate}
		\item \textbf{Находим объединение множеств А и В}
		\begin{enumerate}[{6}. 1.]
		\item Создаем новый вектор
		\item Записываем в вектор первый элемент массива А
		\item Добавляем в вектор следующий элемент массива А
		\item Повторяем пункт 6. 3. до тех пор, пока в вектор не будет записан весь массив А
		\item Берем первый элемент массива B, сравниваем со всеми элементами вектора. Если нет ни одного схожего с ним элемента, то записываемего в вектор, иначе переходим к следующему элементу массива B
		\item Повторяем пункт 6. 5. для остальных элементов массива B
		\end{enumerate}
		\item \textbf{Сортируем вектор объединения множеств A и B}
		\begin{enumerate}[{7}. 1.]
		\item Выполняем алгоритм поаналогии с пунктами 5. 1. и 5. 2.
		\end{enumerate}
		\item \textbf{Находим разность множеств A и B}
		\begin{enumerate}[{8}. 1.]
		\item Создаем вектор
		\item Сравниваем первый элемент массива А со всеми элементами массива B. Если совпадений не находится, то записываем элемент массива A в вектор, иначе переходим к следующему элементу А
		\item Повторяем пункт 8. 2. до тех пор, пока элементы в массиве А не кончатся
		\end{enumerate}
		\item \textbf{Находим разность множеств B и A}
		\begin{enumerate}[{9}. 1.]
			\item Создаем вектор
			\item Сравниваем первый элемент массива B со всеми элементами массива A. Если совпадений не находится, то записываем элемент массива B в вектор, иначе переходим к следующему элементу B
			\item Повторяем пункт 8. 2. до тех пор, пока элементы в массиве B не кончатся
		\end{enumerate}
		\item \textbf{Находим симметрическую разность A и B}
		\begin{enumerate}[{10}. 1.]
			\item создаем вектор
			\item Записываем в вектор первый элемент вектора разности A/B
			\item Добавляем в вектор следующий элемент вектора разности A/B
			\item Повторяем пункт 10. 3. до тех пор, пока в вектор не будет записан весь вектор разности A/B
			\item Записываем в вектор первый элемент вектора разности B/A
			\item Добавляем в вектор следующий элемент вектора разности B/A
			\item Повторяем пункт 10. 6. до тех пор, пока в вектор не будет записан весь вектор разности B/A
		\end{enumerate}
		\item \textbf{Находим пересечение}
		\begin{enumerate}[{11}. 1.]
			\item Создаем вектор
			\item Сравниваем первый элемент вектора объединения со всеми элементами вектора симметричной разности. Если совпадений не находится, то записываем элемент вектора объединения в новый вектор, иначе переходим к следующему элементу вектора объединения
			\item Повторяем пункт 11. 2. до тех пор, пока элементы в векторе симметричной разности не кончатся
		\end{enumerate}
		\item \textbf{Находим дополнение A}
		\begin{enumerate}[{11}. 1.]
			\item Создаем вектор
			\item Сравниваем первый элемент вектора универсума со всеми элементами массива A. Если совпадений не находится, то записываем элемент универсума в вектор, иначе переходим к следующему элементу универсума
			\item Повторяем пункт 11. 2. до тех пор, пока элементы в векторе универсума не кончатся
		\end{enumerate}
		\item \textbf{Вывод массивов}
		\begin{enumerate}[{12.} 1.]
			\item выводим первый элемент массива
			\item выводисм следующий элемент массива
			\item повторяем пункт 12. 2. до тех пор, пока номер элемента не станет равным размеру массива
		\end{enumerate}
		\item \textbf{Вывод векторов}
		\begin{enumerate}[{12.} 1.]
			\item выводим первый элемент вектора
			\item выводисм следующий элемент вектора
			\item повторяем пункт 13. 2. до тех пор, пока номер элемента не станет равным размеру вектор
		\end{enumerate}
	\end{enumerate}
\end{document}